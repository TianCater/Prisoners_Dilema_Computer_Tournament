\documentclass[11pt,preprint, authoryear]{elsarticle}

\usepackage{lmodern}
%%%% My spacing
\usepackage{setspace}
\setstretch{1.2}
\DeclareMathSizes{12}{14}{10}{10}

% Wrap around which gives all figures included the [H] command, or places it "here". This can be tedious to code in Rmarkdown.
\usepackage{float}
\let\origfigure\figure
\let\endorigfigure\endfigure
\renewenvironment{figure}[1][2] {
    \expandafter\origfigure\expandafter[H]
} {
    \endorigfigure
}

\let\origtable\table
\let\endorigtable\endtable
\renewenvironment{table}[1][2] {
    \expandafter\origtable\expandafter[H]
} {
    \endorigtable
}


\usepackage{ifxetex,ifluatex}
\usepackage{fixltx2e} % provides \textsubscript
\ifnum 0\ifxetex 1\fi\ifluatex 1\fi=0 % if pdftex
  \usepackage[T1]{fontenc}
  \usepackage[utf8]{inputenc}
\else % if luatex or xelatex
  \ifxetex
    \usepackage{mathspec}
    \usepackage{xltxtra,xunicode}
  \else
    \usepackage{fontspec}
  \fi
  \defaultfontfeatures{Mapping=tex-text,Scale=MatchLowercase}
  \newcommand{\euro}{€}
\fi

\usepackage{amssymb, amsmath, amsthm, amsfonts}

\def\bibsection{\section*{References}} %%% Make "References" appear before bibliography


\usepackage[round]{natbib}

\usepackage{longtable}
\usepackage[margin=2.3cm,bottom=2cm,top=2.5cm, includefoot]{geometry}
\usepackage{fancyhdr}
\usepackage[bottom, hang, flushmargin]{footmisc}
\usepackage{graphicx}
\numberwithin{equation}{section}
\numberwithin{figure}{section}
\numberwithin{table}{section}
\setlength{\parindent}{0cm}
\setlength{\parskip}{1.3ex plus 0.5ex minus 0.3ex}
\usepackage{textcomp}
\renewcommand{\headrulewidth}{0.2pt}
\renewcommand{\footrulewidth}{0.3pt}

\usepackage{array}
\newcolumntype{x}[1]{>{\centering\arraybackslash\hspace{0pt}}p{#1}}

%%%%  Remove the "preprint submitted to" part. Don't worry about this either, it just looks better without it:
\makeatletter
\def\ps@pprintTitle{%
  \let\@oddhead\@empty
  \let\@evenhead\@empty
  \let\@oddfoot\@empty
  \let\@evenfoot\@oddfoot
}
\makeatother

 \def\tightlist{} % This allows for subbullets!

\usepackage{hyperref}
\hypersetup{breaklinks=true,
            bookmarks=true,
            colorlinks=true,
            citecolor=blue,
            urlcolor=blue,
            linkcolor=blue,
            pdfborder={0 0 0}}


% The following packages allow huxtable to work:
\usepackage{siunitx}
\usepackage{multirow}
\usepackage{hhline}
\usepackage{calc}
\usepackage{tabularx}
\usepackage{booktabs}
\usepackage{caption}


\newenvironment{columns}[1][]{}{}

\newenvironment{column}[1]{\begin{minipage}{#1}\ignorespaces}{%
\end{minipage}
\ifhmode\unskip\fi
\aftergroup\useignorespacesandallpars}

\def\useignorespacesandallpars#1\ignorespaces\fi{%
#1\fi\ignorespacesandallpars}

\makeatletter
\def\ignorespacesandallpars{%
  \@ifnextchar\par
    {\expandafter\ignorespacesandallpars\@gobble}%
    {}%
}
\makeatother

\newlength{\cslhangindent}
\setlength{\cslhangindent}{1.5em}
\newenvironment{CSLReferences}%
  {\setlength{\parindent}{0pt}%
  \everypar{\setlength{\hangindent}{\cslhangindent}}\ignorespaces}%
  {\par}


\urlstyle{same}  % don't use monospace font for urls
\setlength{\parindent}{0pt}
\setlength{\parskip}{6pt plus 2pt minus 1pt}
\setlength{\emergencystretch}{3em}  % prevent overfull lines
\setcounter{secnumdepth}{5}

%%% Use protect on footnotes to avoid problems with footnotes in titles
\let\rmarkdownfootnote\footnote%
\def\footnote{\protect\rmarkdownfootnote}
\IfFileExists{upquote.sty}{\usepackage{upquote}}{}

%%% Include extra packages specified by user
\usepackage{colortbl}
\usepackage{booktabs}
\usepackage{longtable}
\usepackage{array}
\usepackage{multirow}
\usepackage{wrapfig}
\usepackage{float}
\usepackage{colortbl}
\usepackage{pdflscape}
\usepackage{tabu}
\usepackage{threeparttable}
\usepackage{threeparttablex}
\usepackage[normalem]{ulem}
\usepackage{makecell}
\usepackage{xcolor}\usepackage{booktabs}
\usepackage{longtable}
\usepackage{array}
\usepackage{multirow}
\usepackage{wrapfig}
\usepackage{float}
\usepackage{colortbl}
\usepackage{pdflscape}
\usepackage{tabu}
\usepackage{threeparttable}
\usepackage{threeparttablex}
\usepackage[normalem]{ulem}
\usepackage{makecell}
\usepackage{xcolor}

%%% Hard setting column skips for reports - this ensures greater consistency and control over the length settings in the document.
%% page layout
%% paragraphs
\setlength{\baselineskip}{12pt plus 0pt minus 0pt}
\setlength{\parskip}{12pt plus 0pt minus 0pt}
\setlength{\parindent}{0pt plus 0pt minus 0pt}
%% floats
\setlength{\floatsep}{12pt plus 0 pt minus 0pt}
\setlength{\textfloatsep}{20pt plus 0pt minus 0pt}
\setlength{\intextsep}{14pt plus 0pt minus 0pt}
\setlength{\dbltextfloatsep}{20pt plus 0pt minus 0pt}
\setlength{\dblfloatsep}{14pt plus 0pt minus 0pt}
%% maths
\setlength{\abovedisplayskip}{12pt plus 0pt minus 0pt}
\setlength{\belowdisplayskip}{12pt plus 0pt minus 0pt}
%% lists
\setlength{\topsep}{10pt plus 0pt minus 0pt}
\setlength{\partopsep}{3pt plus 0pt minus 0pt}
\setlength{\itemsep}{5pt plus 0pt minus 0pt}
\setlength{\labelsep}{8mm plus 0mm minus 0mm}
\setlength{\parsep}{\the\parskip}
\setlength{\listparindent}{\the\parindent}
%% verbatim
\setlength{\fboxsep}{5pt plus 0pt minus 0pt}



\begin{document}



\begin{frontmatter}  %

\title{Beyond Tit-for-Tat: A repeated prisoner's dilemma computer
tournament}

% Set to FALSE if wanting to remove title (for submission)




\author[Add1]{Joshua Strydom\footnote{17 August 2022}}
\ead{20718284@sun.ac.za}

\author[Add1]{Tian Cater}
\ead{19025831@sun.ac.za}

\author[Add1,Add2]{Sven Wellman}
\ead{20850980@sun.ac.za}



\address[Add1]{Advanced Microeconomics First Proposal: Group 2}



\vspace{1cm}





\vspace{0.5cm}

\end{frontmatter}



%________________________
% Header and Footers
%%%%%%%%%%%%%%%%%%%%%%%%%%%%%%%%%
\pagestyle{fancy}
\chead{}
\rhead{}
\lfoot{}
\rfoot{\footnotesize Page \thepage}
\lhead{}
%\rfoot{\footnotesize Page \thepage } % "e.g. Page 2"
\cfoot{}

%\setlength\headheight{30pt}
%%%%%%%%%%%%%%%%%%%%%%%%%%%%%%%%%
%________________________

\headsep 35pt % So that header does not go over title




\hypertarget{the-roles-assigned-to-each-member}{%
\section{The Roles Assigned To Each
Member}\label{the-roles-assigned-to-each-member}}

Through a collective effort, our group has managed to program the code
for the computer tournament fully. Even though the exact specification
of the game's rules (for example, number of iterations, number of
strategies, imperfect information regarding number of iterations etc.)
has not yet been finalised, our model's flexibility allows for
effortless adjustment of the rules.

In the next phase, all members will collectively investigate the
intricacies surrounding different strategies under divergent
circumstances to narrow our research question and potentially submit a
competitive strategy of our own design for the tournament.

\hypertarget{what-is-the-prisoners-dilemma}{%
\section{What is the Prisoner's
Dilemma?}\label{what-is-the-prisoners-dilemma}}

A prisoner's dilemma is a non-zero-sum game in which players attempt to
maximise their advantage without concern for the well-being of the other
players. Each player has two choices, either cooperate or defect. The
players have no way of communicating their intentions. In equilibrium,
each prisoner chooses to defect even though the joint payoff would be
higher by cooperating. Unfortunately for the prisoners, each is
incentivised to cheat even after promising to cooperate. This is the
heart of the dilemma.

\hypertarget{our-preliminary-plan}{%
\section{Our (Preliminary) Plan}\label{our-preliminary-plan}}

This paper conducts a computer tournament to study effective choice in
the iterated prisoner's dilemma game. Drawing from Axelrod
(\protect\hyperlink{ref-axelrod1980}{1980})'s computer tournament, we
will distinguish between two types of games, one in which the number of
rounds played is known by each player (or strategy) beforehand, (Round
1) and the other where it is determined probabilistically, effectively
purging some minor end-game effects (Round 2). Both games allow for
mutual gains from cooperation and possible exploitation of a strategy by
another strategy. A preliminary observation is that there is no one best
strategy. Table \ref{Figure1} provides a summary of some of the most
competitive strategies that have been played.

\begin{table}

\caption{\label{tab:unnamed-chunk-1}Some strategies played in past computer tournaments \label{Figure1}}
\centering
\begin{threeparttable}
\begin{tabular}[t]{>{}l>{\raggedright\arraybackslash}p{30em}}
\toprule
Strategy & Description\\
\midrule
\textbf{Tit-for-Tat (TFT)} & Begins by cooperating and then simply repeats the last moves made by the opponent.\\
\textbf{Generous TFT} & Same as TFT but 'forgives' defections in 1/3 of cases.\\
\textbf{Tit-for-Two-Tat} & Like TFT but only retaliates after two defections rather than one.\\
\textbf{DOWNING} & Based on outcome maximization principle, if other player is responsive to DOWNING, it will cooperate, if not, will defect.\\
\textbf{JOSS} & This strategy plays Tit For Tat, always defecting if the opponent defects but cooperating when the opponent cooperates with probability .9.\\
\addlinespace
\textbf{AllC} & Always cooperates.\\
\textbf{AllD} & Always defects.\\
\textbf{Alternate} & Randomly cooperate or defect (prob = 1/2) on the 1st round then alternates regardless of what the opponent does.\\
\textbf{Grudger} & Cooperates until the opponent defects and then defects forever.\\
\textbf{Random} & This strategy plays randomly (prob = 1/2) disregarding the history of play.\\
\addlinespace
\textbf{Detective} & Cooperates, defects, cooperates and cooperates again. If the opponent doesn't relatilates in the 3rd round, defects all the time;otherwise plays Tit-for-Tat.\\
\textbf{Win-Stay-Lose-Shift} & Cooperates first then, if the opponent cooperated on the last round, repeat last move; otherwise, switch.\\
\bottomrule
\end{tabular}
\begin{tablenotes}
\item \textit{Note: } 
\item There are many more prominent strategies that have proven to be effective, and will be considered in the final simulation of the tournament.
\end{tablenotes}
\end{threeparttable}
\end{table}

For Round 1, in which the number of rounds is known, the programmed
strategies play against each other. Each program has information on the
history of past interactions and can make choices based on this. Axelrod
(\protect\hyperlink{ref-axelrod1980}{1980})`s computer tournament found
that the properties of `being nice' and `being forgiving' positively
influences the utility points `scored' by each strategy. A `friendly'
means that a strategy is not the first to defect before the last few
moves, and being forgiving means that a process has a propensity to
cooperate after another strategy playing defect. Axelrod \& others
(\protect\hyperlink{ref-axelrod1987evolution}{1987}) found defection to
be the best strategy when playing the prisoner's dilemma with a fixed,
finite, and known number of rounds. This is because there are end game
effects by which players have an incentive to defect to gain an extra
payoff. The Tit-for-Tat strategy came out on top as the winner. These
results are, however, not definitive.

For Round 2, the end game effects will be effectively eliminated as the
game's length was determined probabilistically. The players will be
assumed to know which strategies performed well in Round 1. In addition,
each program has information on the history of past plays and analysis
of Round 1. As in Round 1, the `being nice' property positively
correlates with how well a strategy performed. Again, in Axelrod
(\protect\hyperlink{ref-axelrod1980}{1980})'s tournament, the
Tit-for-Tat strategy came out on top as the winner. There is said to be
no best rule independent of the environment. Tit-for-Tat is a very
robust strategy as it combines being nice, retaliatory, forgiving and
clear.

\newpage

\hypertarget{references}{%
\section*{References}\label{references}}
\addcontentsline{toc}{section}{References}

\hypertarget{refs}{}
\begin{CSLReferences}{1}{0}
\leavevmode\vadjust pre{\hypertarget{ref-axelrod1980}{}}%
Axelrod, R. 1980. Effective choice in the prisoner's dilemma.
\emph{Journal of conflict resolution}. 24(1):3--25.

\leavevmode\vadjust pre{\hypertarget{ref-axelrod1987evolution}{}}%
Axelrod, R. et al. 1987. The evolution of strategies in the iterated
prisoner's dilemma. \emph{The dynamics of norms}. 1:1--16.

\end{CSLReferences}

\bibliography{Tex/ref}





\end{document}
